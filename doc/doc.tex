% \input utf8-t1
\documentclass[11pt,a4paper,titlepage]{extarticle}

\usepackage[english]{babel}
\usepackage[utf8]{inputenc}
\usepackage{fancyhdr}
\usepackage[obeyspaces]{url}
\usepackage[paper=a4paper,top=1.5cm,left=1.5cm,right=1.5cm,bottom=1.5cm]{geometry}
\usepackage{listings}
\usepackage{graphicx}
\usepackage{enumitem}
\usepackage{subcaption}
\usepackage{float}
\setcounter{secnumdepth}{1}
\setlength{\parindent}{0pt}
\setlength{\parskip}{0.5\bigskipamount}
\usepackage{amsmath} % for \text
\usepackage{tikz}
\usetikzlibrary{automata,positioning}

\lhead{}
\chead{}
\rhead{}
\lfoot{}
\cfoot{}

\begin{document}
\pagestyle{empty}
\begin{center}
	\section*{UPA: Ukládání a příprava dat -- 1. část projektu}
\end{center}

\large{Zvolené téma: \textbf{Databáze meteorologických dat}}

\large{
	Řešitelé:
	Bc. Josef Kolář (\textit{xkolar71}),
	Bc. Timotej Halás (\textit{xhalas10}),
	Bc. Vojtěch Hertl (\textit{xhertl04})
}

\section{Zvolené dotazy a formulace vlastního dotazu}

\begin{itemize}
	\item[\textbf{A}] vytvořte žebříček nejdeštivějších/nejsušších a nejteplejších/nejchladnějších meteorologických stanic/lokalit
	\item[\textbf{B}] najděte skupiny meteorologických stanic s podobným počasím
	\item[\textbf{C}] \ldots
\end{itemize}
\section{Stručná charakteristika zvolené datové sady}
(Zde konkrétně popište jaké soubory budou představovat zdroj dat pro zvolené úlohy. Dále popište, jakým způsobem budou tato data získána a stručně charakterizujte strukturu souborů vybraných pro řešení projektu. Zaměřte se na části souborů, které jsou důležité pro zodpovězení zvolených dotazů.)

\begin{lstlisting}[language=XML]
<station wmo-id="94648"
         stn-name="ADELAIDE (WEST TERRACE / NGAYIRDAPIRA)" type="AWS"
         lat="-34.9257" lon="138.5832" stn-height="29.32"
         forecast-district-id="SA_PW001"
         description="Adelaide (West Terrace /  ngayirdapira)"
    >

    <period index="0" time-utc="2020-09-26T21:50:00+00:00" wind-src="OMD">
        <level index="0" type="surface">
            <element units="Celsius" type="apparent_temp">6.4</element>
            <element type="cloud">Partly cloudy</element>
            <element type="cloud_oktas">4</element>
            <element units="Celsius" type="delta_t">2.1</element>
            <element units="km/h" type="gust_kmh">9</element>
            <element units="knots" type="wind_gust_spd">5</element>
            <element units="Celsius" type="air_temperature">9.0</element>
            <element units="Celsius" type="dew_point">4.2</element>
            <element units="hPa" type="pres">1027.4</element>
            <element units="hPa" type="msl_pres">1027.4</element>
            <element units="hPa" type="qnh_pres">1027.4</element>
            <element units="%" type="rel-humidity">72</element>
            <element units="km" type="vis_km">61</element>
            <element type="wind_dir">ENE</element>
            <element units="deg" type="wind_dir_deg">69</element>
            <element units="km/h" type="wind_spd_kmh">7</element>
            <element units="knots" type="wind_spd">4</element>
        </level>
    </period>
</station>
\end{lstlisting}

\section{Zvolený způsob uložení uložení surových dat}
V rámci této databáze se bude jednat o kolekce station a measurement. Kolekce station obsahuje základní informace o stanici provozující měření. Kolekce measurement obsahuje všechny vykonané měření a na stanici, která vykonala měření se odkazuje pomocí identifikátoru stanice. Detailní popis obsahu kolekcí:
\begin{itemize}
	\item \textbf{station}
	\begin{itemize}[label=\textperiodcentered]
    	\item \texttt{wmo\_id} -- unikátní identifikátor stanice
        \item \texttt{location} -- stát a město kde se nachází stanice
        \item \texttt{station\_name} -- název stanice
        \item \texttt{station\_height} -- nadmořská výška, ve které se stanice nachází
        \item \texttt{latitude}, \texttt{longitude} -- zeměpisná poloha stanice
    \end{itemize}
	\item \textbf{measurement}
	\begin{itemize}[label=\textperiodcentered]
        \item \texttt{station} -- odkaz na unikátní identifikátor stanice
        \item \texttt{time\_period} -- čas měření
        \item \texttt{delta\_t} -- indikátor rychlosti vypařování
        \item \texttt{dew\_point} -- teplota, na kterou musí být vzduch zchlazený aby vznikla kondenzace
        \item \texttt{rel\_humidity} -- vlhkost vzduch
        \item \texttt{vis\_km} -- viditelnost v kilometrech
        \item \texttt{weather} -- typ počasí slovně
        \item \texttt{pres}, \texttt{msl\_pres}, \texttt{qnh\_press} -- údaje popisující atmosferický tlak
        \item \texttt{rain\_hour}, \texttt{rain\_ten} -- počet srážek v milimetrech
        \item \texttt{air\_temperature}, \texttt{apparent\_temp} -- pocitová a reální teplota vzduchu
        \item \texttt{cloud}, \texttt{cloud\_oktas}, \texttt{cloud\_type\_id} -- typ oblaků, jejich pokrytí oblohy a slovní popis oblačnosti
        \item \texttt{wind\_dir}, \texttt{wind\_dir\_deg} -- směr větru popsán slovně a úhlem 
        \item \texttt{wind\_spd\_kmh}, \texttt{wind\_spd} -- rychlost větru v kilometrech za hodinu a v uzlech získán průměrem za 10 minut
        \item \texttt{gust\_kmh}, \texttt{wind\_gust\_spd} --rychlost větru v kilometrech za hodinu a v uzlech získán ze 3 sekund měření
        \item \texttt{rainfall}, \texttt{rainfall\_24hr} -- počet srážek v milimetrech od 9:00 a historický údaj z posledního dne
        \item \texttt{minimum\_air\_temperature}, \texttt{maximum\_air\_temperature} -- minimální a maximální naměřená teplota od 18:00 do 9:00
        \item \texttt{maximum\_gust\_kmh}, \texttt{maximum\_gust\_spd}, \texttt{maximum\_gust\_dir} -- maximální naměřená rychlost větru v kilomterech za hodinu a uzlech a jeho směr od půlnoci do půlnoci z 3 sekundových měření
    \end{itemize}
\end{itemize}
\end{document}
