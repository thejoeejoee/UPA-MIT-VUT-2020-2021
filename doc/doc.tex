% \input utf8-t1
\documentclass[10pt,a4paper,titlepage]{extarticle}

\usepackage[english]{babel}
\usepackage[utf8]{inputenc}
\usepackage{fancyhdr}
\usepackage[obeyspaces]{url}
\usepackage[paper=a4paper,top=1.5cm,left=1.5cm,right=1.5cm,bottom=1.5cm]{geometry}
\usepackage{listings}
\usepackage{graphicx}
\usepackage{enumitem}
\usepackage{subcaption}
\usepackage{float}
\usepackage{todonotes}
\setcounter{secnumdepth}{1}
\setlength{\parindent}{0pt}
\setlength{\parskip}{0.5\bigskipamount}
\usepackage{amsmath} % for \text
\usepackage{tikz}
\usepackage{hyperref}
\usetikzlibrary{automata,positioning}


\usepackage{color}

\definecolor{codeprimary}{HTML}{3300CC}
\colorlet{keywordstyle}{codeprimary!70!red}
\colorlet{stringstyle}{codeprimary!25!red}
\colorlet{commentstyle}{purple!90!white}

\lstset{
    captionpos=b,
    belowskip=0pt,
}

\lstdefinelanguage{XML}{
    morestring=[b]",
    morestring=[s]{>}{<},
    morecomment=[s]{<?}{?>},
    basicstyle=\color{codeprimary}\ttfamily,
    keywordstyle=\color{keywordstyle}\bfseries,
    commentstyle=\color{commentstyle}\ttfamily,
    stringstyle=\color{stringstyle},
    columns=fullflexible,
    showstringspaces=false,
}
\renewcommand\lstlistingname{Kód}

\hypersetup{
    colorlinks,
    linkcolor={codeprimary},
    citecolor={codeprimary},
    urlcolor={codeprimary}
}

\begin{document}

\begin{center}
	\section*{UPA: Ukládání a příprava dat -- 1. část projektu}
\end{center}

\large{Zvolené téma: \textbf{Databáze meteorologických dat}}

\large{
	Řešitelé:
	Bc. Josef Kolář (\textit{xkolar71}),
	Bc. Timotej Halás (\textit{xhalas10}),
	Bc. Vojtěch Hertl (\textit{xhertl04})
}%
\section{Zvolené dotazy a formulace vlastního dotazu}
\begin{itemize}
	\item[\textbf{A}] vytvořte žebříček nejdeštivějších/nejsušších a nejteplejších/nejchladnějších meteorologických stanic/lokalit
	\item[\textbf{B}] najděte skupiny meteorologických stanic s~podobným počasím
	\item[\textbf{C}] vizualizujte průměrnou teplotu vzduchu na kontinentu Austrálie i případnou interpolací hodnot
    do regionů bez měřících stanic
\end{itemize}%
\section{Stručná charakteristika zvolené datové sady}
Na FTP serveru australského meteorologického úřadu se nachází datově i formátově široká sada mapující počasí po celém
kontinentu Austrálie -- pro účely tohoto projektu, resp. dle jeho zadání, bude k~dalšímu zpracování použit výběr sedmi
datových souborů ve formátu XML.

Jejich sběr ze vzdáleného serveru bude mít na starost samostatný Docker kontejner s~příznačným názvem
\texttt{scraper} -- jeho implementace bude realizována v~jazyce Python a během svého běhu se připojí na FTP server,
detekuje aktualizované soubory od poslední kontroly, ty novější stáhne na lokální úložiště a následně je uloží
do databáze MongoDB. Ta běží v~samostatném kontejneru a její kolekce jsou popsány v~sekci
\nameref{sec:zvoleny-zpusob-ulozeni-surovych-dat}.

Jednou z~důležitých částí datových XML souborů je specifikace stanice, ve které probíhají měření -- ukázka toho
fragmentu datové sady je umístěna níže v~\nameref{lst:station-example}. Mezi atributy důležité pro řešení úloh
v~tomto projektu je především dvojice \mbox{\texttt{[lat, lon]}} určující geografické umístění stanice, unikátní
identifikátor stanice \texttt{wmo-id} (který bude použit pro navázání jednotlivých měření) a samotné jméno stanice
dostupné z~dvojice atributů \texttt{stn-name} a \texttt{description}.

\begin{lstlisting}[language=XML,caption={Ukázka způsobu uložení informací týkajících se konkrétní stanice.},
label=lst:station-example]
<station
    wmo-id="94648" forecast-district-id="SA_PW001"
    stn-name="ADELAIDE (WEST TERRACE / NGAYIRDAPIRA)" type="AWS"
    lat="-34.9257" lon="138.5832" stn-height="29.32"
    description="Adelaide (West Terrace /  ngayirdapira)"
><!-- all measurements from this station based on time --></station>\end{lstlisting}
Obsahem těchto elementů jsou poté elementy typu \texttt{<period />}, jehož atribut \texttt{time-utc} určuje čas
konkrétního měření. Při zanoření do těchto elementů se poté dostáváme přímo ke změřeným datům, která jsou uložena
v~elementech typu \texttt{<element />} -- ty ve všech případech obsahují atribut \texttt{type}, který značí, o~jaký typ
změřených dat se jedná. Samotná data jsou poté uložena jako obsah elementu a jejich jednotka, je-li to nutné, je
doplněna v~atributu \texttt{units}. Krácená varianta tohoto uložení se nachází níže v~kódu
\nameref{lst:measurement-example}.
\begin{lstlisting}[language=XML,caption={Ukázka uložení meteorologických dat změřených v~\textit{21:50:00 UTC 26.9
.2020}}, label=lst:measurement-example]
<period time-utc="2020-09-26T21:50:00+00:00">
    <level type="surface">
        <element units="Celsius" type="apparent_temp">6.4</element>
        <element type="cloud">Partly cloudy</element>
        <element type="cloud_oktas">4</element>
        <element units="Celsius" type="delta_t">2.1</element>
        <element units="km/h" type="gust_kmh">9</element>
        <element units="knots" type="wind_gust_spd">5</element>
        <element units="Celsius" type="air_temperature">9.0</element>
        <element units="hPa" type="pres">1027.4</element>
        <element units="%" type="rel-humidity">72</element>
        <element units="km" type="vis_km">61</element>
        <element type="wind_dir">ENE</element>
        <element units="deg" type="wind_dir_deg">69</element>
        <element units="km/h" type="wind_spd_kmh">7</element>
    </level>
</period>\end{lstlisting}%
\section{Zvolený způsob uložení uložení surových dat}\label{sec:zvoleny-zpusob-ulozeni-surovych-dat}
V~rámci této databáze se bude jednat o~kolekce \texttt{station} a \texttt{measurement}. První jmenovaná ukládá
informace o~samotné meteorologické stanici provozující měření.
Kolekce \texttt{measurement} pak obsahuje všechna vykonaná měření a na stanici,
která vykonala konkrétní měření, se odkazuje pomocí identifikátoru stanice.
\begin{itemize}
	\item \textbf{station}
	\begin{itemize}[label=\textperiodcentered]
    	\item \texttt{wmo\_id} -- unikátní identifikátor stanice
        \item \texttt{location} -- stát a město, kde se stanice nachází
        \item \texttt{station\_name} -- název stanice
        \item \texttt{station\_height} -- nadmořská výška v~metrech, ve které se stanice nachází
        \item \texttt{latitude}, \texttt{longitude} -- zeměpisná poloha stanice
    \end{itemize}
	\item \textbf{measurement}
	\begin{itemize}[label=\textperiodcentered]
        \item \texttt{station} -- odkaz na unikátní identifikátor stanice
        \item \texttt{time\_period} -- čas měření
        \item \texttt{delta\_t} -- indikátor rychlosti vypařování
        \item \texttt{dew\_point} -- teplota, na kterou musí být vzduch zchlazen, aby došlo k~jeho
        kondenzaci\footnote{též tzv. rosný bod -- \url{https://en.wikipedia.org/wiki/Dew\_point}}
        \item \texttt{rel\_humidity} -- relativní vlhkost vzduchu
        \item \texttt{vis\_km} -- viditelnost v~kilometrech
        \item \texttt{weather} -- typ počasí slovně
        \item \texttt{pres}, \texttt{msl\_pres}, \texttt{qnh\_press} -- údaje popisující atmosferický tlak
        \item \texttt{rain\_hour}, \texttt{rain\_ten} -- množství srážek v~milimetrech
        \item \texttt{air\_temperature}, \texttt{apparent\_temp} -- pocitová a reálná teplota vzduchu
        \item \texttt{cloud}, \texttt{cloud\_oktas}, \texttt{cloud\_type\_id} -- typ oblaků, jejich pokrytí oblohy a slovní popis oblačnosti
        \item \texttt{wind\_dir}, \texttt{wind\_dir\_deg} -- směr větru popsán slovně a úhlem 
        \item \texttt{wind\_spd}, \texttt{wind\_spd\_kmh} -- rychlost větru v~uzelch a kilometrech za hodinu získáná průměrem za 10 minut
        \item \texttt{wind\_gust\_spd}, \texttt{gust\_kmh} -- rychlost větru v~uzelch a kilometrech za hodinu získáná
        ze 3sekundových měření
        \item \texttt{rainfall}, \texttt{rainfall\_24hr} -- počet srážek v~milimetrech od ranních 9:00 a historický
        údaj z~předešlého dne
        \item \texttt{minimum\_air\_temperature}, \texttt{maximum\_air\_temperature} -- minimální a maximální naměřená teplota od 18:00 do 9:00
        \item \texttt{maximum\_gust\_spd}, \texttt{maximum\_gust\_kmh}, \texttt{maximum\_gust\_dir} -- maximální
        naměřená rychlost větru v~uzlech a kilometrech za hodinu a jeho směr od půlnoci do půlnoci z~3sekundových měření
    \end{itemize}
\end{itemize}


\section{Zvolený SQL model pro uložení relační databáze}

\todo{nejlépe vytvořit diagram databáze/možná ER, ale možná to tady vůbec nemusí být}

\section{Implementace projektu a workflow}

Projekt byl rozdělen do několika částí - mongo, scraper, postgres, computer, redis a superset. Implementace jednotlivých částí projektu byla propojena softwarem Docker. Každá z vyjmenovaných částí byla reprezentována pomocí odpovídajícího Docker kontejneru a tedy běží jako samotný podprogram a všechny dohromady tvoří celou aplikaci. V následující sekci budou popsány kontejnery a jejich funkcionalita.

\subsection{mongo}

Kontejner mongo vytvoří instanci MongoDB a vytvoří uživatele pro kontejner \textbf{scraper} a \textbf{computer}. K mongu patří jeden kontejner navíc, který je určen pro administrátora.

\subsection{scraper}

Scraper závisí na kontejneru mongo a slouží ke stažení dat z určeného serveru, k parsování získaných dat do vyhovujícího formátu a uložení do databáze. Přesněji se scraper nejprve připojí na FTP server s požadovanými meteorologickými daty  ve formátu XML a stáhne je. Alternativně může také data načíst z lokálního úložiště. Následně data naparsuje do formátu uvedeného v sekci \ref{sec:zvoleny-zpusob-ulozeni-surovych-dat}, který je definován pomocí Mongo dokumentů a nakonec je uloží do instance \textbf{MongoDB}.


\subsection{postgres}

Tento kontejner vytvoří instanci SQL relační databáze PostgreSQL a v ní vytvoří databázi a uživatele s různými rolemi a právy. S postgresem má souvisí další dva kontejnery, které je určen pro administrátora - jeden je postgres administrátor a druhý django administrátor, který se stará o Django modely v postgresu.

\subsection{computer}

Computer závisí na kontejnerech mongo a postgres, mezi kterými synchronizuje meteorologické stanice z datábází mongo do postgresu Především však vykonává dotazy pomocí operací nad databázemi tím způsobem, aby byly splněny body A a B ze zadání.

Podrobnější popis dotazů:

\paragraph{Dotaz A:}

\todo{možná popsat pipeliny těch dotazů - lookup -> project -> group atd..}

\paragraph{Dotaz B:}


\subsection{redis}

Kontejner redis vytvoří instanci Redis.


\subsection{superset}

Superset závisí na kontejnerech postgres a redis, se kterými vyměnuje data. Vytvoří instanci vizualizační platformy Superset a propojí s úložištěm redis, který používá jako cache.

\end{document}
